%%
%% This is file `sample-sigconf.tex',
%% generated with the docstrip utility.
%%
%% The original source files were:
%%
%% samples.dtx  (with options: `sigconf')
%% 
%% IMPORTANT NOTICE:
%% 
%% For the copyright see the source file.
%% 
%% Any modified versions of this file must be renamed
%% with new filenames distinct from sample-sigconf.tex.
%% 
%% For distribution of the original source see the terms
%% for copying and modification in the file samples.dtx.
%% 
%% This generated file may be distributed as long as the
%% original source files, as listed above, are part of the
%% same distribution. (The sources need not necessarily be
%% in the same archive or directory.)
%%
%%
%% Commands for TeXCount
%TC:macro \cite [option:text,text]
%TC:macro \citep [option:text,text]
%TC:macro \citet [option:text,text]
%TC:envir table 0 1
%TC:envir table* 0 1
%TC:envir tabular [ignore] word
%TC:envir displaymath 0 word
%TC:envir math 0 word
%TC:envir comment 0 0
%%
%%
%% The first command in your LaTeX source must be the \documentclass
%% command.
%%
%% For submission and review of your manuscript please change the
%% command to \documentclass[manuscript, screen, review]{acmart}.
%%
%% When submitting camera ready or to TAPS, please change the command
%% to \documentclass[sigconf]{acmart} or whichever template is required
%% for your publication.
%%
%%
\documentclass[sigconf,10pt, screen]{acmart}

%%
%% \BibTeX command to typeset BibTeX logo in the docs
\AtBeginDocument{%
  \providecommand\BibTeX{{%
    Bib\TeX}}}

%% Rights management information.  This information is sent to you
%% when you complete the rights form.  These commands have SAMPLE
%% values in them; it is your responsibility as an author to replace
%% the commands and values with those provided to you when you
%% complete the rights form.
\setcopyright{none}
\copyrightyear{2023}
\acmYear{2023}
\acmDOI{XXXXXXX.XXXXXXX}

%% These commands are for a PROCEEDINGS abstract or paper.
\acmConference[SysSem'23]{Systems Seminar (SysSem)}{Feb-Mar 2023}{Amsterdam, the Netherlands}

%%
%% end of the preamble, start of the body of the document source.
\begin{document}

%% The "title" command has an optional parameter,
%% allowing the author to define a "short title" to be used in page headers.
\title{Artifact Evaluation report on [Paper Title] (max 6-pages)}
% \subtitle{Subtitle Text, if any\titlenote{with optional subtitle note}}
\author{Your Name}
\affiliation{%
  \institution{Student number: 0000000}
  \country{}
}
\begin{abstract}
Put a couple of lines regarding the summary of your key findings. 
\end{abstract}

\settopmatter{printfolios=true}
\maketitle

\section{Artifact Evaluation Process}
Give a short summary of the key claims of the paper, and attached artifact. Give your motivation of what do you think are the key contributions of the work are, and what \textbf{three key experiments} are you planning to reproduce. Please justify your choice of choosing the three key experiments. 
\subsection{Three Key Experiments}
\begin{itemize}
    \item \textbf{Experiment 1:} Which figure, which experiment from the paper and why.
    \item \textbf{Experiment 2:} Which figure, which experiment from the paper and why.
    \item \textbf{Experiment 3:} Which figure, which experiment from the paper and why.
\end{itemize}

You goal for this report is to provide as detailed explanation of how you conducted the artifact evaluation. \textbf{Do not forget to study the accompanying guide on what is expected in this report.}

%%%%%%%%%%%%%%%%%%%%%%%%%%%%%%%%%%%%%%%%%%%%%%%%%%%%%%%%%%%%%%%%%%%%%
\section{Evaluation Workflow}

In order to fill our the following section, please study about The CTuning project (\url{https://ctuning.org/ae/}) and the following link~\cite{2023-ae-checklist}. Further context regarding reproducibility research is discussed in~\cite{2017-cacm-notes-repro,2020-sigcomm-rev-badging,2023-sigops-ae-notes,2023-sigplan-ae-guidelines}. ACM badging guidelines~\cite{2023-acm-badge}.

ACM also has a new special interest group (SIG), \url{https://reproducibility.acm.org/}.

This checklist comes from \url{https://github.com/ctuning/ck-artifact-evaluation/blob/master/wfe/artifact-evaluation/templates/ae.tex}.
\subsection{Artifact check-list (meta-information)}

{\small
\begin{itemize}
  \item {\bf Algorithm: }
  \item {\bf Program: }
  \item {\bf Compilation: }
  \item {\bf Transformations: }
  \item {\bf Binary: }
  \item {\bf Model: }
  \item {\bf Data set: }
  \item {\bf Run-time environment: }
  \item {\bf Hardware: }
  \item {\bf Run-time state: }
  \item {\bf Execution: }
  \item {\bf Metrics: }
  \item {\bf Output: }
  \item {\bf Experiments: }
  \item {\bf How much disk space required (approximately)?: }
  \item {\bf How much time is needed to prepare workflow (approximately)?: }
  \item {\bf How much time is needed to complete experiments (approximately)?: }
  \item {\bf Publicly available?: }
  \item {\bf Code licenses (if publicly available)?: }
  \item {\bf Data licenses (if publicly available)?: }
  \item {\bf Workflow framework used?: }
  \item {\bf Archived (provide DOI)?: }
\end{itemize}
}

%%%%%%%%%%%%%%%%%%%%%%%%%%%%%%%%%%%%%%%%%%%%%%%%%%%%%%%%%%%%%%%%%%%%%
\subsection{Description}

\subsubsection{How to access}

\subsubsection{Hardware dependencies}

\subsubsection{Software dependencies}

\subsubsection{Data sets}

\subsubsection{Models}

%%%%%%%%%%%%%%%%%%%%%%%%%%%%%%%%%%%%%%%%%%%%%%%%%%%%%%%%%%%%%%%%%%%%%
\subsection{Installation}

%%%%%%%%%%%%%%%%%%%%%%%%%%%%%%%%%%%%%%%%%%%%%%%%%%%%%%%%%%%%%%%%%%%%%
\subsection{Experiment workflow}

%%%%%%%%%%%%%%%%%%%%%%%%%%%%%%%%%%%%%%%%%%%%%%%%%%%%%%%%%%%%%%%%%%%%%
\subsection{Evaluation and expected results}


%%%%%%%%%%%%%%%%%%%%%%%%%%%%%%%%%%%%%%%%%%%%%%%%%%%%%%%%%%%%%%%%%%%%%
\subsection{Experiment customization}

%%%%%%%%%%%%%%%%%%%%%%%%%%%%%%%%%%%%%%%%%%%%%%%%%%%%%%%%%%%%%%%%%%%%%
\subsection{Notes}

%%%%%%%%%%%%%%%%%%%%%%%%%%%%%%%%%%%%%%%%%%%%%%%%%%%%%%%%%%%%%%%%%%%%%
\subsection{Methodology}

Submission, reviewing and badging methodology. Which of the three ACM badges will you give to this paper and justify. ACM badges: \url{https://www.acm.org/publications/policies/artifact-review-badging}

\bibliographystyle{ACM-Reference-Format}
\bibliography{main}
% The bibliography should be embedded for final submission.

\end{document}